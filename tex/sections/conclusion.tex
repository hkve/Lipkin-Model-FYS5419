A simple 2x2 Hamiltonian consisting of a interacting and non-interacting part has been modelled and the lowest energy of this Hamiltonian has been compared to the results of the VQE algorithm for varying degrees of interaction strength, ranging from $\lambda \in [0,1]$. VQE is very well equipped for this problem, yielding satisfying results. For a two qubit system, the Hamiltonian becomes more complex. The lowest energy of the Hamiltonian \Comment{skriv om resultatet, har det ikke tilgjengelig.}.
\newline For the 4x4 Hamiltonian of the two qubit system the entanglement entropy was found to be increasing as a result of larger connection strength. The reduced density matrices of the system is less pure and under interaction and will create a strong entanglement when the energy of interaction surpasses that of the non-interaction Hamiltonian. 
\Comment{Skrive noe om Qiskit, og at vi har blitt flinke i QC? (som faktisk er målet her)}
For the Lipkin model with $W=0$ we find good correspondence when using the $RY$ ansatz for the two qbit schemes, outperforming HF and RPA for two and four particles across a range of interaction strengths. Using the $Q = N$ scheme for four particles yielded more variability, demonstrating that reducing complexity of Pauli strings and encoding schemes not only reduce computational time but also increase precision. Considering $W, V > 0$, low to medium interaction strengths up to $\sim \epsilon/3$ outperformed RPA and HF. For highly interactive systems, convergence to good ground state energies was more difficult. Exploring the dependency on different types of ansatz choices could improve the results for stronger interactions.
