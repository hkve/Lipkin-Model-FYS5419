\text{balbla..}

For the Lipkin model with $W=0$ we find good correspondence when using the $RY$ ansatz for the two qbit schemes, outperforming HF and RPA for two and four particles across a range of interaction strengths. Using the $Q = N$ scheme for four particles yielded more variability, demonstrating that reducing complexity of Pauli strings and encoding schemes not only reduce computational time but also increase precision. Considering $W, V > 0$, low to medium interaction strengths up to $\sim \epsilon/3$ outperformed RPA and HF. For highly interactive systems, convergence to good ground state energies was more difficult. Exploring the dependency on different types of ansatz choices could improve the results for higher interactions.