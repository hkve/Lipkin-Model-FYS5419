This section will introduce the three relevant systems for this work. Initially we will look at two `toy models`, being $2\times2$ and $4 \times 4$ real Hamiltonians with arbitrary entries. Lastly the Lipkin model will be used, showing many of the key features of many-body systems while still being analytically solvable.   

\subsection{$H \in \mathbb{R}^2$ Hamiltonian}
As an initial test, we will consider a simply $2\times 2$ real Hamiltonian consistend of a diagonal part $H_0$ and off-diagonal part $H_I$, playing the roles of a non-interactive one-body and interactive two-body part respectively. Defined through their matrix elements, we express them in the Pauli basis $\set{\ket{0},\ket{1}}$

\begin{align}
    \begin{split} \label[eq]{eq:theo:2times2_hamiltonian}
        H &= H_0 + H_I \\
        H_0 = \begin{pmatrix}
            E_1 & 0 \\
            0 & E_2
        \end{pmatrix}&, \hspace{20px}
        H_I = \lambda \begin{pmatrix}
            V_{11} & V_{12} \\
            V_{21} & V_{22}
        \end{pmatrix}
    \end{split}
\end{align}
Where $\lambda \in [0,1]$ is a coupling constant parameterizing the strength of the interaction. 

\subsection{$H \in \mathbb{R}^4$ Hamiltonian}
We now move on to a slightly larger system, defined as a $4 \times 4$ real Hamiltonian. This can be viewed as two  composite systems where each system is a two-level system. In the product basis $\set{\ket{00},\ket{01},\ket{10},\ket{11}}$ we define the one-body part as 
\begin{align}
    H_0 \ket{ij} = \epsilon_{ij}\ket{ij}, \label[eq]{eq:theo:4times4_diagonal_hamiltonian}
\end{align}
with a two-body interaction defined using Pauli matricies
\begin{align}
    \begin{split} \label[eq]{eq:theo:4times4_interaction}
        H_I &= H_x X \otimes X + H_z Z \otimes Z \\
        &= \begin{pmatrix}
            H_z & 0 & 0 & H_z \\
            0 & - H_z & H_x & 0 \\
            0 & H_x & - H_z & 0 \\
            H_x & 0 & 0 & H_z
        \end{pmatrix}
    \end{split}
\end{align}
Where $H_x$ and $H_y$ are couplings playing the same role as $\lambda$.

\subsection{Lipkin Model}
For larger many-body systems, the introduction of the occupation representation is common. The creation of a state $p$ is represented by the operator $\crt{p}$, while annihilation $\ani{p}$. Since we are dealing with fermions, the creation and annihilation operators follow the canonical anitcommutation relations
\begin{align}
    \set{\crt{p},\crt{q}} = \set{\ani{p},\ani{q}} = 0, \hspace{10px}\set{\crt{p},\ani{q}} = \delta_{pq}. 
    \label[eq]{eq:theo:car_algebra}
\end{align}
The Lipkin model \citep{lipkinValidityManybodyApproximation1965} for a $N$ fermion system consists of two energy levels $\sigma \in \set{\pm 1}$, each having a degeneracy of $N$. Resembling fermionic half integer spin, the two levels contributes energetically $\pm \frac{1}{2}\epsilon$ for each particle. Additionally, we label the different states in each degenerate level by $p = 1, \ldots N$ \footnote{This removes the spin quantum number from $p$, which commonly is included.}. The Lipkin Hamiltonian can be written using creation and annihilation operators as 
\begin{align}
    \begin{split}\label[eq]{eq:theo:lipkin_second_quant_hamiltonian}
    H &= H_0 + H_1 + H_2, \\
    H_0 &= \frac{1}{2}\epsilon \sum_{pq} \sigma \crt{p\sigma} \ani{p\sigma}, \\
    H_1 &= \frac{1}{2}V \sum_{pp'\sigma} \crt{p\sigma}\crt{p'\sigma}\ani{p'\bar{\sigma}}\ani{p\bar{\sigma}}, \\
    H_2 &= \frac{1}{2}W \sum_{pp'\sigma} \crt{p\sigma} \crt{p'\bar{\sigma}} \ani{p'\sigma} \ani{p\bar{\sigma}}.
    \end{split}
\end{align}
With $\bar{\sigma} = -\sigma$. $H_0$ is a diagonal single particle operator giving a contribution based on occupancy in the different levels. $H_1$ and $H_2$ are two-body operators moving pairs of particles between levels and exchanging particles between levels (spin-exchange) respectively. The strengths of these effects are parameterized by the couplings $V$ and $W$. By defining the \textit{quasi-spin} operators $J_\pm , J_z, J^2$ in addition to the number operator $N$ as 
\begin{align}
    \begin{split}\label[eq]{eq:theo:quasi_spin_definitions}
        J_\pm = \sum_p \crt{p\pm}\ani{p\mp} \\
        J_z = \frac{1}{2}\sum_{p\sigma} \sigma \crt{p\sigma} \ani{p\sigma} \\
        J^2 = J_+ J_- + J_z^2 - J_z \\
        N = \sum_{p\sigma} \crt{p\sigma} \ani{p\sigma}
    \end{split}
\end{align}
we can rewrite the Lipkin Hamiltonian from \cref{eq:theo:lipkin_second_quant_hamiltonian} using these
\begin{align}
    \begin{split} \label[eq]{eq:theo:lipkin_quasispin_hamiltonian}
        H_0 &= \epsilon J_z, \\
        H_1 &= \frac{1}{2}V (J_+^2 + J_-^2), \\
        H_2 &= \frac{1}{2}W ( \set{J_+ , J_- } - N ),
    \end{split}
\end{align}
These quasi spin operators obey the normal spin commutator relations
\begin{align*}
    &\commutator{J_z}{J_\pm} = \pm J_\pm, 
    &\commutator{J_+}{J_-} = 2 J_z, \\
    &\commutator{J^2}{J_\pm} = 0,
    &\commutator{J^2}{J_z} = 0,
\end{align*}
in addition to commuting with the number operator
\begin{align*}
    \commutator{N}{J_z} = \commutator{N}{J_\pm} = \commutator{N}{J^2} = 0.
\end{align*}
Using these relations we can show that the Hamiltonian (a product of quasi spin operators and the number operator) also commutes with  $J^2$, that is $\commutator{H}{J^2}=0$. Therefor $H$ and $J^2$ have shared eigenbasis, with $J$ being a `good` quantum number.

Using spin-eigenstates as the Hamiltonian basis, we define states through the normal approach $\ket{J,J_z}$ with $J$ and $ J_z$ as spin and spin-projections respectively. The states $J_z = \pm J$ are the easiest to construct, corresponding to a single level being completely filled. States in between can then be found using the quasi-spin ladder operators following

\begin{align}
    J_\pm \ket{J,J_z} &= \sqrt{J(J+1)-J_z (J_z \pm 1)} \ket{J,J_z \pm 1}. \label[eq]{eq:theo:ladder_operations}
\end{align}
Using this basis for the quasi-spin Hamiltonian from \cref{eq:theo:lipkin_quasispin_hamiltonian}, the explicit matrix $H_{J_z,J'_z} = \bra{J,J_z} H \ket{J,J_z'}$ can be constructed. For $N=2$ particles, we have the triplet $J=1$, with three possible projection $J_z = 0, \pm 1$. As shown in \cref{sec:app:lipkin_spin_one_hamiltonian}, the Hamiltonian can be written in the $J$ basis as
\begin{align}
    H = \begin{pmatrix} \label[eq]{eq:theo:lipkin_N2_matrix}
        -\epsilon & 0 & V \\
        0 & W & 0 \\
        V & 0 & \epsilon
    \end{pmatrix}.
\end{align}
By solving the eigenvalue problem, the ground state energy can be exactly found. Similarly, for the $N=4$ case we have $J = 2$ with five possible projections $J_z = 0,\pm1,\pm2$. Using the same approach as in \cref{sec:app:lipkin_spin_one_hamiltonian}, the Hamiltonian can be expressed as  
\begin{align}
    H = \begin{pmatrix}\label[eq]{eq:theo:lipkin_N4_matrix}
        -2\epsilon & 0  & \sqrt{6}V &0 &0 \\
        0 & -\epsilon + 3W & 0 & 3V & 0 \\
        \sqrt{6}V & 0 & 4W & 0 & \sqrt{6}V \\
        0 & 3V & 0 & \epsilon + 3W & 0 \\
        0 & 0 & \sqrt{6}V & 0 & 2\epsilon
    \end{pmatrix}.
\end{align}
To benchmark the VQE results for the Lipkin Model we will also compare with results from Hartree-Fock (HF) and random phase approximation (RPA). For the Lipkin Model these are also exactly solvable. From \citep{coHartreeFockRandom2015} we have the Hartree-Fock solution
\begin{align}
    \EHF = -\frac{N}{2}\begin{cases}
        \epsilon + W & R < \epsilon \\
        \frac{\epsilon^2 + (N-1)^2 (V+W)^2 }{2(N-1)(V+W)} + W & R > \epsilon,
    \end{cases} \label[eq]{eq:theo:hartreefock_lipkin}
\end{align}
which is split into two regions based on the relative strength between the interactions $V,W$ and the single particle energies $\epsilon$, $R \equiv (N-1)(V+W)$. Also from \citep{coHartreeFockRandom2015}, we have the RPA solution
\begin{align}
    \ERPA = \EHF + \frac{\omega - A}{2},
\end{align}
where
\begin{align*}
    \omega = \sqrt{A^2 - |B|^2}.
\end{align*}
The coefficients $A$ and $B$ are region specific, explicitly

\begin{align*}
    A &= \begin{cases}
        \epsilon - (N-1)W & R < \epsilon \\
        \frac{3(N-1)^2(V+W)^2 - \epsilon^2}{2(N-1)(V+W)} - (N-1)W & R > \epsilon,
    \end{cases} \\
    B &= \begin{cases}
        -(N-1)V & R < \epsilon \\
        -\frac{\epsilon^2 + (N-1)^2 (V+W)^2 }{2(N-1)(V+W)} + (N-1)W & R > \epsilon.
    \end{cases}
\end{align*}

\subsection{Encodings}
\subsubsection{Toy Hamiltonians}
Firstly we need to express the Hamiltonian from \cref{eq:theo:2times2_hamiltonian} using Pauli matrices. Beginning with the diagonal $H_0$, we need to fill the both entries with different energy values. Defining their difference using

\begin{align*}
    E_+ = \frac{E_1 + E_2}{2},\hspace{20px} E_- = \frac{E_1 - E_2}{2}
\end{align*}
we see that by combining the identity and Z Pauli matrix, this can be expressed as

\begin{align*}
    H_0 = E_+ I + E_- Z
\end{align*}
For $H_1$ we use the same trick to fill the diagonal, defining $V_+ = (V_{11} + V_{22})/2, V_- = (V_{11} - V_{22})/2$. From the hermiticity requirements of $H$, we note that $V_{12} = V_{21} \equiv V_o$, which simplifies the problem to a simple $X$. Therefor

\begin{align*}
    H_I = V_+ I + V_- Z + V_o X
\end{align*}

Show explicit $2\times 2$ and $4\times 4$ expressions with Pauli matrices.
\subsubsection{Lipkin Model}
Using the level mapping from \citep{hlatshwayoSimulatingExcitedStates2022}, the spin operators from \cref{eq:theo:quasi_spin_definitions} can be written using their one-body counterparts 
\begin{align}
    J_z = \sum_{i}^{N} \spinsmall{z}{i} \hspace{20px} J_\pm = \sum_i^N \spinsmall{\pm}{i} = \sum_i^N (\spinsmall{x}{i}\pm i\spinsmall{y}{i}) \label[eq]{eq:met:quasispin_to_onebody_mapping}
\end{align}
with $N$ being the number of particles. Additionally, since we have spin-$\sfrac{1}{2}$ fermions, the mapping to Pauli matrices follow
\begin{align}
    \spinsmall{x}{i} = \frac{1}{2}X_i,\hspace{20px}\spinsmall{y}{i} = \frac{1}{2}Y_i,\hspace{20px}\spinsmall{z}{i} = \frac{1}{2}Z_i. \label[eq]{eq:met:onebody_to_pauli_mapping}
\end{align}
As shown in \cref{sec:app:pauliencoding_detailed}, \cref{eq:theo:lipkin_quasispin_hamiltonian} can be expressed as
\begin{align}
    \begin{split}\label[eq]{eq:met:lipkin_pauli_hamiltonian} 
        H_0 &= \frac{\epsilon}{2}\sum_{p}Z_p,  \\
        H_1 &= \frac{1}{2}V \sum_{p < q} X_p X_q - Y_p Y_q, \\
        H_2 &= \frac{1}{2}W \sum_{p < q} \pclosed{ X_p X_q + Y_p Y_q }. 
    \end{split}
\end{align}
We will investigate the integer spin systems $N=2\hspace{5px}(J=1)$ and $N=4\hspace{5px}(J=2)$. Writing out the expressions from \cref{eq:met:lipkin_pauli_hamiltonian} we have for two particles
\begin{align}
    H^{N=2} = \frac{\epsilon}{2}\pclosed{Z_1 + Z_2} + \frac{W+V}{2} X_1 X_2 - \frac{W-V}{2} Y_1 Y_2 \label[eq]{eq:met:lipkin_N4_dumb}
\end{align}
and for four particles
\begin{align}
    \begin{split} \label[eq]{eq:met:lipkin_N4_dumb}
        H^{N=4} =& \frac{\epsilon}{2}\pclosed{Z_1 + Z_2 + Z_3 + Z_4} + \frac{W-V}{2}\Big( X_1 X_2 +\\ 
        & X_1 X_3 + X_1 X_4 + X_2 X_3 + X_3 X_4 + X_3 X_4
        \Big) \\
        &+ \frac{W+V}{2}\Big( Y_1 Y_2 + Y_1 Y_3 + Y_1 Y_4 + Y_2 Y_3 \\
        & + Y_3 Y_4 + Y_3 Y_4\Big)
    \end{split}
\end{align}
Which is quite a mouthful. If we set $W = 0$, reductions of the problem can be performed. Looking back at \cref{eq:theo:lipkin_quasispin_hamiltonian}, we note that if only $H_0$ and $H_1$ are included,  spins differing by $\pm2$ are the only possible non-diagonal couplings. Since $H_0$ is diagonal and single particle energies are degenerate, it does not break the pairing symmetry. Instead of the $2J+1$ spin projections, we simply have $J+1$ relevant states. From \citep{hlatshwayoSimulatingExcitedStates2022} we have the two-qbit Hamiltonian for $N=4$ as
\begin{align}
    H_{W=0}^{N=4} = \epsilon(Z_1 + Z_2) + \frac{\sqrt{6}}{2}V (X_1 + X_2 + Z_1 X_0 - X_1 Z_0). \label[eq]{eq:met:lipkin_N4_smart_hamiltonian}
\end{align}
This will be more computationally efficient due to requiring only half the qbits of \cref{eq:met:lipkin_N4_dumb}. Both sche

\section{Calculating density matrices}
Given a Hamiltonian consisting of a non- and an interacting Hamiltonian on a two qubit state, a state will have eigenvectors with coefficients $\alpha_{ij}$. For a Hamiltonian, which in matrix form has $4\times 4$ dimension, there will be four sets of eigenvectors, given a representation of the states in the form
\begin{align}
    \ket{00} = \begin{pmatrix}1& 0 & 0 & 0
    \end{pmatrix}^T \\ 
    \ket{01} = \begin{pmatrix}0& 1 & 0 & 0
    \end{pmatrix}^T \\
    \ket{10} = \begin{pmatrix}0& 0 & 1 & 0
    \end{pmatrix}^T \\
    \ket{11} = \begin{pmatrix}0& 0 & 0 & 1
    \end{pmatrix}^T 
\end{align}
The resulting density matrix, will be the outer product of each of these eigenstates of the Hamiltonian. The density matrix can be decomposed into partial density matrices describing the separate Hilbert space of one of the qubits, by partially tracing over the other qubit. This is done in the below calculation.
\begin{align}
        &\rho = \alpha_{00} \ket{00}\bra{00} + \alpha_{01} \ket{01}\bra{01} + \alpha_{10} \ket{10}\bra{10} + \alpha_{11} \ket{11}\bra{11} \\ 
        \rho_A = T&r_B(\rho)  = \big(\boldsymbol{I}\otimes \bra{\psi}\big)\, \rho \big( \boldsymbol{I}\otimes \ket{\psi}\big) \quad \rho_B = Tr_A(\rho)  = \big(\bra{\psi}\otimes \boldsymbol{I}\big)\, \rho \big(\ket{\psi}\otimes \boldsymbol{I}\big) 
\end{align}
With the state $\ket{\psi} = \frac{1}{\sqrt{2}} (\ket{0} + \ket{1})$, the resulting reduced density matrix $\rho_A$ and $\rho_B$ is
\begin{align}
    \rho_A &= \frac{1}{2} \Big( \alpha_{00} \ket{0}\bra{0} + \alpha_{01} \ket{0}\bra{0} + \alpha_{10} \ket{1}\bra{1} + \alpha_{11} \ket{1}\bra{1}    \Big) \\ 
    &= \frac{1}{2} \Big(\big(\alpha_{00}+\alpha_{01}\big)\ket{0}\bra{0} + \big(\alpha_{10}+\alpha_{11}\big)\ket{1}\bra{1}\Big) \\
    \rho_B &= \frac{1}{2 } \Big( \alpha_{00} \ket{0}\bra{0} + \alpha_{01} \ket{1}\bra{1} + \alpha_{10} \ket{0}\bra{0} + \alpha_{11} \ket{1}\bra{1}\Big)\\
    &= \frac{1}{2} \Big(\big(\alpha_{00}+\alpha_{10}\big)\ket{0}\bra{0} + \big(\alpha_{01}+\alpha_{11}\big)\ket{1}\bra{1}\Big)
\end{align}
The von-Neumann entanglement entropy is defined by the trace of the product between the density matrix and the logarithm of the density matrix, as follows
\begin{align}
    S(\rho_i) &= - Tr\Big(\rho_i \ln \rho_i\Big) \\
    \implies S(\rho_a)& = \frac{\ln2}{2}\Big(\alpha_{00}+\alpha_{10}\big) \times \big(\alpha_{01}+\alpha_{11}\big)\Big)\\ \times&\ln\big(\alpha_{00}+\alpha_{10}\big) \times\ln \big(\alpha_{01}+\alpha_{11}\big) \nonumber
\end{align}
It quantizes the entanglement between the two separate Hilbert spaces. The maximal entropy for a two qubit mixed state, will be $\ln 2$, that of a Bell state. This leads to a possible visualization of the entanglement von Neumann entropy as a metric of the probability of achieving a Bell state, given a mixed state, normmalized by $\ln 2$.
