
\subsection{Hamiltonians}
\subsubsection{Lipkin Model}
For larger many-body systems, the introduction of the occupation representation is common. The creation of a state $p$ is represented by the operator $\crt{p}$, while annihilation $\ani{p}$. Since we are dealing with fermions, the creation and annihilation operators follow the canonical anitcommutation relations
\begin{align}
    \set{\crt{p},\crt{q}} = \set{\ani{p},\ani{q}} = 0, \hspace{10px}\set{\crt{p},\ani{q}} = \delta_{pq}. 
    \label[eq]{eq:theo:car_algebra}
\end{align}
The Lipkin model Hamiltonian can be written using creation and annihilation operators as 
\begin{align}
    \begin{split}\label[eq]{eq:theo:lipkin_second_quant_hamiltonian}
    H &= H_0 + H_1 + H_2, \\
    H_0 &= \frac{1}{2}\epsilon \sum_{pq} \sigma \crt{p\sigma} \ani{p\sigma}, \\
    H_1 &= \frac{1}{2}V \sum_{pp'\sigma} \crt{p\sigma}\crt{p'\sigma}\ani{p'\bar{\sigma}}\ani{p\bar{\sigma}}, \\
    H_2 &= \frac{1}{2}W \sum_{pp'\sigma} \crt{p\sigma} \crt{p'\bar{\sigma}} \ani{p'\sigma} \ani{p\bar{\sigma}}.
    \end{split}
\end{align}
With $\bar{\sigma} = -\sigma$. By defining the \textit{quasi-spin} operators $J_\pm , J_z, J^2$ in addition to the number operator $N$ as 
\begin{align}
    \begin{split}\label[eq]{eq:theo:quasi_spin_definitions}
        J_\pm = \sum_p \crt{p\pm}\ani{p\mp} \\
        J_z = \frac{1}{2}\sum_{p\sigma} \sigma \crt{p\sigma} \ani{p\sigma} \\
        J^2 = J_+ J_- + J_z^2 - J_z \\
        N = \sum_{p\sigma} \crt{p\sigma} \ani{p\sigma}
    \end{split}
\end{align}
we can rewrite the Lipkin Hamiltonian from \cref{eq:theo:lipkin_second_quant_hamiltonian} using these
\begin{align}
    \begin{split} \label[eq]{eq:theo:lipkin_quasispin_hamiltonian}
        H_0 &= \epsilon J_z, \\
        H_1 &= \frac{1}{2}V (J_+^2 + J_-^2), \\
        H_2 &= \frac{1}{2}W ( \set{J_+ , J_- } - N ),
    \end{split}
\end{align}
These quasi spin operators obey the normal spin commutator relations
\begin{align*}
    &\commutator{J_z}{J_\pm} = \pm J_\pm, 
    &\commutator{J_+}{J_-} = 2 J_z, \\
    &\commutator{J^2}{J_\pm} = 0,
    &\commutator{J^2}{J_z} = 0,
\end{align*}
in addition to commuting with the number operator
\begin{align*}
    \commutator{N}{J_z} = \commutator{N}{J_\pm} = \commutator{N}{J^2} = 0.
\end{align*}
Using these relations we can show that the Hamiltonian (a product of quasi spin operators and the number operator) also commutes with  $J^2$, that is $\commutator{H}{J^2}=0$. Therefor $H$ and $J^2$ have shared eigenstates, with $J$ being a `good` quantum number.