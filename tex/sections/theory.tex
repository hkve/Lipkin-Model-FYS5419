
\subsection{Hamiltonians}
\subsubsection{Lipkin Model}
For larger many-body systems, the introduction of the occupation representation is common. The creation of a state $p$ is represented by the operator $\crt{p}$, while annihilation $\ani{p}$. Since we are dealing with fermions, the creation and annihilation operators follow the canonical anitcommutation relations
\begin{align}
    \set{\crt{p},\crt{q}} = \set{\ani{p},\ani{q}} = 0, \hspace{10px}\set{\crt{p},\ani{q}} = \delta_{pq}. 
    \label[eq]{eq:theo:car_algebra}
\end{align}
The Lipkin model \citep{lipkinValidityManybodyApproximation1965} for a $N$ fermion system consists of two energy levels $\sigma \in \set{\pm 1}$, each having a degeneracy of $N$. Resembling fermionic half integer spin, the two levels contributes energetically $\pm \frac{1}{2}\epsilon$ for each particle. Additionally, we label the different states in each degenerate level by $p = 1, \ldots N$ \footnote{This removes the spin quantum number from $p$, which commonly is included.}. The Lipkin Hamiltonian can be written using creation and annihilation operators as 
\begin{align}
    \begin{split}\label[eq]{eq:theo:lipkin_second_quant_hamiltonian}
    H &= H_0 + H_1 + H_2, \\
    H_0 &= \frac{1}{2}\epsilon \sum_{pq} \sigma \crt{p\sigma} \ani{p\sigma}, \\
    H_1 &= \frac{1}{2}V \sum_{pp'\sigma} \crt{p\sigma}\crt{p'\sigma}\ani{p'\bar{\sigma}}\ani{p\bar{\sigma}}, \\
    H_2 &= \frac{1}{2}W \sum_{pp'\sigma} \crt{p\sigma} \crt{p'\bar{\sigma}} \ani{p'\sigma} \ani{p\bar{\sigma}}.
    \end{split}
\end{align}
With $\bar{\sigma} = -\sigma$. $H_0$ is a diagonal single particle operator giving a contribution based on occupancy in the different levels. $H_1$ and $H_2$ are two-body operators moving pairs of particles between levels and exchanging particles between levels (spin-exchange) respectively. The strengths of these effects are parameterized by the couplings $V$ and $W$. By defining the \textit{quasi-spin} operators $J_\pm , J_z, J^2$ in addition to the number operator $N$ as 
\begin{align}
    \begin{split}\label[eq]{eq:theo:quasi_spin_definitions}
        J_\pm = \sum_p \crt{p\pm}\ani{p\mp} \\
        J_z = \frac{1}{2}\sum_{p\sigma} \sigma \crt{p\sigma} \ani{p\sigma} \\
        J^2 = J_+ J_- + J_z^2 - J_z \\
        N = \sum_{p\sigma} \crt{p\sigma} \ani{p\sigma}
    \end{split}
\end{align}
we can rewrite the Lipkin Hamiltonian from \cref{eq:theo:lipkin_second_quant_hamiltonian} using these
\begin{align}
    \begin{split} \label[eq]{eq:theo:lipkin_quasispin_hamiltonian}
        H_0 &= \epsilon J_z, \\
        H_1 &= \frac{1}{2}V (J_+^2 + J_-^2), \\
        H_2 &= \frac{1}{2}W ( \set{J_+ , J_- } - N ),
    \end{split}
\end{align}
These quasi spin operators obey the normal spin commutator relations
\begin{align*}
    &\commutator{J_z}{J_\pm} = \pm J_\pm, 
    &\commutator{J_+}{J_-} = 2 J_z, \\
    &\commutator{J^2}{J_\pm} = 0,
    &\commutator{J^2}{J_z} = 0,
\end{align*}
in addition to commuting with the number operator
\begin{align*}
    \commutator{N}{J_z} = \commutator{N}{J_\pm} = \commutator{N}{J^2} = 0.
\end{align*}
Using these relations we can show that the Hamiltonian (a product of quasi spin operators and the number operator) also commutes with  $J^2$, that is $\commutator{H}{J^2}=0$. Therefor $H$ and $J^2$ have shared eigenbasis, with $J$ being a `good` quantum number.

Using spin-eigenstates as the Hamiltonian basis, we define states through the normal approach $\ket{J,J_z}$ with $J$ and $ J_z$ as spin and spin-projections respectively. The states $J_z = \pm J$ are the easiest to construct, corresponding to a single level being completely filled. States in between can then be found using the quasi-spin ladder operators following

\begin{align}
    J_\pm \ket{J,J_z} &= \sqrt{J(J+1)-J_z (J_z \pm 1)} \ket{J,J_z \pm 1}. \label[eq]{eq:theo:ladder_operations}
\end{align}
Using this basis for the quasi-spin Hamiltonian from \cref{eq:theo:lipkin_quasispin_hamiltonian}, the explicit matrix $H_{J_z,J'_z} = \bra{J,J_z} H \ket{J,J_z'}$ can be constructed.

\comment{Rekkefølgen følger $\bra{J,-J} H \ket{J, -J}$ som $H_{0,0}$, motsatt av artikkelen}

\comment{Her med $W$, tror Morten har surra litt og her brukt hamiltonen $H = H_0 - H_1 - H_2$ samt uten $N$ operator}
\begin{align}
    H = \begin{pmatrix}
        -(\epsilon-W) & 0 & -V \\
        0 & -2W & 0 \\
        -V & 0 & \epsilon - W
    \end{pmatrix}
\end{align}

\comment{med $H = H_0 + H_1 + H_2$ og $N$ inkludert}
\begin{align}
    H = \begin{pmatrix}
        -\epsilon & 0 & V \\
        0 & W & 0 \\
        V & 0 & \epsilon
    \end{pmatrix}
\end{align}

\comment{$J=2$ er good med $H = H_0 + H_1 + H_2$}
\begin{align}
    H = \begin{pmatrix}
        -2\epsilon & 0  & \sqrt{6}V &0 &0 \\
        0 & -\epsilon + 3W & 0 & 3V & 0 \\
        \sqrt{6}V & 0 & 4W & 0 & \sqrt{6}V \\
        0 & 3V & 0 & \epsilon + 3W & 0 \\
        0 & 0 & \sqrt{6}V & 0 & 2\epsilon
    \end{pmatrix}
\end{align}

\subsection{Pauli Encoding Hamiltonians}
To apply a specific system to VQE, the Hamiltonian must be re-written in terms of a sum of \textit{Pauli strings}. Considering the identity $I$ and the three Pauli matrices $X, Y, Z$, we construct a specific term as a tensor product of these $P_i$. For an $n$-qubit system, we require $n-1$ tensor products. Combined with a weight $w_i$, the Hamiltonian must be written in the from
\begin{align}
    H = \sum_i w_i P_i \label[eq]{eq:theo:pauli_encoding_general}
\end{align}
For a general two-body Hamiltonian, this task can serve challenging. Many schemes exists for this purpose, such as the Jordan-Wigner transformation \citep{steudtnerMethodsSimulateFermions2019}. For the Lipkin model, an easier approach is available since the Hamiltonian can be written as products of the number and spin operators.