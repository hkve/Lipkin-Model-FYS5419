\section{$J=1$ Lipkin Hamiltonian}\label[sec]{sec:app:lipkin_spin_one_hamiltonian}
Since we have expressed the Hamiltonian using quasi-spin operators, investigating how the states $\ket{J, J_z}$ are related to each other will be beneficial. For $J=1$ ($N=2$) we can write the $J_z = \pm 1$ states as

\begin{align*}
    \ket{1,\pm 1} = \crt{1\pm}\crt{2\pm}\ket{0}
\end{align*}
The states are related to each other, in additon to the third $\ket{1,0}$ states through \cref{eq:theo:ladder_operations}.

\begin{align*}
    J_+ \ket{1,-1} &= \sqrt{2}\ket{1,0}\hspace{20px}&J_- \ket{1,-1} &= 0 \\
    J_+ \ket{1,0} &= \sqrt{2}\ket{1,1} &J_- \ket{1,0} &= \sqrt{2}\ket{1,-1} \\
    J_+ \ket{1,1} &= 0  &J_- \ket{1,1} &= \sqrt{2}\ket{1,0}
\end{align*}
Inspecting \cref{eq:theo:lipkin_quasispin_hamiltonian} we see that $H_0$ only acts on the diagonal
\begin{align*}
    \bra{J,J_z} H_0 \ket{J,J_z'} = \epsilon J_z \delta_{J_z,J_z'}
\end{align*} 
In addition, the pair mover, as the name entails, only if there are states differing by two units of angular momentum since $H_1$ contains $J_\pm^2$. We see that

\begin{align*}
    \bra{J,J_z} H_1 \ket{J,J_z \pm 2} &= \frac{1}{2}V \bra{J,J_z} J_\mp^2 \ket{J,J_z \pm 2} \\
    &= \frac{1}{2}V (\sqrt{2})^2 \bra{J,J_z}\ket{J,J_z} = V
\end{align*}
Lastly $H_2$ have $J_\pm J_\mp$  and $N$ terms contributing for all diagonal terms. For $\ket{1,\pm 1}$, these cancel out since

\begin{align*}
    \bra{1,\pm 1} J_\pm J_\mp \ket{1,\pm 1} &= 2 \\
    \bra{1,\pm 1} N \ket{1,\pm 1} &= 2
\end{align*}
But for $\ket{1,0}$ we get a contribution from both $J_+J_-$ and $J_-J_+$, giving a total contribution off

\begin{align*}
    \bra{1,0} H_2 \ket{1,0} &= \frac{1}{2}W\bra{1,0} J_+ J_- + J_- J_+ - N \ket{1,0} \\
    &= \frac{1}{2}W (2 + 2 - 2) = W
\end{align*}
Combining these results we find 

\begin{align*}
    H = \begin{pmatrix}
        -\epsilon & 0 & V \\
        0 & W & 0 \\
        V & 0 & \epsilon
    \end{pmatrix}
\end{align*}
Ordered such that $\bra{J,J_z} H \ket{J,J_z'} = H_{J_z+1,J_z'+1}$

\section{Pauli Encoding}\label[sec]{sec:app:pauliencoding_detailed}
We will now convert \cref{eq:theo:quasi_spin_definitions} to \cref{eq:met:lipkin_pauli_hamiltonian} using the spin mappings from \cref{eq:met:quasispin_to_onebody_mapping} and \cref{eq:met:onebody_to_pauli_mapping}. The diagonal term $H_0$ is quite simple, where we simply substitute in for the $z$ spin giving

\begin{align*}
    H_0 = \epsilon J_z = \epsilon \sum_{p} \spinsmall{z}{p} = \frac{\epsilon}{2} \sum_p Z_p
\end{align*}
Moving on to the $H_1$ term, we need to expand the square of the $J_\pm$ operators. Starting with $J_+$ we find
\begin{align}
    J_+^2 &= \pclosed{\sum_p \spinsmall{x}{p} + i\spinsmall{y}{p}}^2 = \sum_{pq}\pclosed{\spinsmall{x}{p} + i\spinsmall{y}{p}}\pclosed{\spinsmall{x}{q} + i\spinsmall{y}{q}} \nonumber\\
    &= \sum_{pq} \spinsmall{x}{p}\spinsmall{x}{q} - \spinsmall{y}{p}\spinsmall{y}{q} + i\spinsmall{y}{p}\spinsmall{x}{q} + i\spinsmall{x}{p}\spinsmall{y}{q} \label[eq]{eq:app:jpluss_square_expanded}
\end{align}
Similarly, the $J_-^2$ term yields
\begin{align}
    J_-^2 = \sum_{pq} \spinsmall{x}{p}\spinsmall{x}{q} - \spinsmall{y}{p}\spinsmall{y}{q} - i\spinsmall{y}{p}\spinsmall{x}{q} - i\spinsmall{x}{p}\spinsmall{y}{q}. \label[eq]{eq:app:jminus_square_expanded}
\end{align}
By inspection we see that the cross terms of \cref{eq:app:jpluss_square_expanded} and \cref{eq:app:jminus_square_expanded} cancel each other. We can then expand in diagonal and off-diagonal terms and substitute in the mapping from \cref{eq:met:onebody_to_pauli_mapping}.
\begin{align*}
    J_+^2 + J_-^2 &= 2\sum_{pq} \spinsmall{x}{p}\spinsmall{x}{q} - \spinsmall{z}{p}\spinsmall{z}{q} \\
    &= 2 \sum_p (\spinsmall{x}{p})^2 - (\spinsmall{y}{p})^2 + 4\sum_{p > q} \spinsmall{x}{p}\spinsmall{x}{q} - \spinsmall{y}{p}\spinsmall{y}{q} \\
    &= \frac{1}{2} \sum_p X_p^2 - Y_p^2 + \sum_{p > q} X_p X_q - Y_p Y_q
\end{align*}
The diagonal term will cancel out, since the Pauli matrices are involutory, giving the pair moving Hamiltonian
\begin{align*}
    H_1 = \frac{1}{2}V \sum_{p > q} X_p X_q - Y_p Y_q 
\end{align*}
\comment{Can do same for $H_2$ to include $W$ but does not get correct result, will have to check the math before texing.}